\documentclass{article}

\title{\vspace{-5mm}%
	\fontsize{24pt}{10pt}\selectfont
	\textbf{VISU- Zusammenfassung}
}	
\usepackage{enumitem}
\usepackage[utf8]{inputenc}

\begin{document}
\maketitle	
\pagebreak
\section{Aufgaben der Visualisierung}
Begriffsdefinition: Drei Zitate zu Visualisierung\\
Aufgaben der Visualisierung Zusammenfassung\\
Aufgaben der Visualisierung I - III\\
Schritte Idealisiertes Vorgehen\\
Ein Wort zu VTK
\pagebreak

\section{Datenformate und Gittermodelle}
\subsection{Eigenschaften von Daten und Datensätzen - Was bilden wir ab?} 
\textbf{Aufbau Datensatz}\\

\begin{itemize}[noitemsep]
	\item Definitionsmenge
	\item Topologie
	\item Funktion $Definitionsmenge \rightarrow Bildmenge$ wobei Bildmenge Attributen der Daten gleichen muss.
\end{itemize}


\noindent \textbf{5 Kriterien von Datensätzen:}
\begin{itemize}[noitemsep]
	\item ....
\end{itemize}

\noindent Typen von Daten:
\begin{itemize}[noitemsep]
	\item ....
\end{itemize}
Nominal bis Kontinuierlich
\begin{itemize}[noitemsep]
	\item ....
\end{itemize}

\noindent \textbf{Anforderungen ans Visualisierungssystem}
\begin{itemize}[noitemsep]
	\item effiziente Daten-Speicherung
	\item effizienter Datenzugriff
	\item Abbildbarkeit (?)
	\item Einfachheit (?)
\end{itemize}

\noindent \textbf{Begriffsdefinitionen}\\
\begin{itemize}[noitemsep]
	\item Stetigkeit (Kurz in Umgangssprache aufschreiben)
	\item Topolgie, Kodimension...
\end{itemize}

\noindent \textbf{Das Problem, dass alles was gemessen wird, diskrete Werte sind}\\
Infos auf Folie?\\
Wie man zwischen Werten interpoliert.\\


\noindent Shepard Interpolation\\

\noindent Modifizierte Shepard Interpolation\\
$\rightarrow$ Mit Radius

\subsection{Interpolation generell - Alles im Zwischenraum visualisieren können }
Zwischen welchen Werten man interpoliert.\\

\noindent \textbf{Auf der Linie}\\

\noindent \textbf{Im linearen Dreieck}\\

\noindent \textbf{Dreieck mit Baryzentrischen Koordinaten}\\

\noindent \textbf{Im linearen Tetraeder}\\
- mit Baryzentrischen Koordinaten\\
- mit linearem Ansatz\\

\noindent \textbf{Im bilinearen Rechteck}\\
- Entlang der Kanten linear\\
- Innerhalb des Rechtecks Bi-Linear\\

\noindent \textbf{Im linearen Quader}\\

\noindent \textbf{Im linearen Prisma}\\

\noindent \textbf{Neares Neighbour in Voroni-Diagrammen für Scattered Data}\\

\subsection{Gitter - den gesamten Raum visualisieren können}
\noindent \textbf{Was ist ein Gitter?}\\
- Bisher Set an Interpolations-Möglichkeiten\\
- Jetzt ganzen Raum mit Gitter auskleiden - befähigt uns jeden Punkt im Raum abzubilden.\\

\noindent \textbf{Welche Gitter gibt es?}\\

\noindent \textbf{Wie speichert man die Dinger?}\\
- Feste X und Y Delta\\
- Suchstrukturen nennen können. (R-Baum, weitere..)
\pagebreak

\section{Skalare und Skalarfelder}

\subsection{Definifion Feld und Skalarfeld}
Feld: ...\\
Skalarfeld...

\subsection{Darstellungsformen von 2D-Definitionsmenge}
\begin{itemize}[noitemsep]
	\item Einfärbung
	\item Höhenfelder
	\item Isolinien
\end{itemize}

\subsection{Darstellungsformen von 3D-Definitionsmenge}
\begin{itemize}[noitemsep]
	\item Schnittflächen mit Einfärbung: Volumen mit einer Ebene schneiden, dann Punkten Farbwert zuordnen.
	\item Jedem Skalar im Volumen optische Eigenschaft zuweisen und auf Projektionsfläche abbilden (Absorptions-, Emissions-Modell, ect.)
	\item Isoflächen (Zeige alle Punkte mit Isowert a an)
\end{itemize}

\subsection{Color-Maps}
Aspekte raussuchen.

\subsection{Isolinien und Flächen}
- Was sind das?\\
- Interpolation von Isowert in Dreiecks-Flächen (Wo schneidet ein Dreieck die 0-Höhenlinie?)\\
- Interpolation von Isowert in Vierecks-Flächen\\
- Interpolation von Isowert in Polygone -> In Drei-oder Vierecke zerlegen, dann weiter\\

\subsection{Isolinien und Flächen im Raum: Marching Cubes}
- Prinzip\\
- Bug im Verfahren (Bei Verbindung über mehrere Zellen)\\
- Korrekte Triangulierung\\ 
- Konsistente Triangulierung -> Über Festlegung: Bei Sattelpunkt: Liegt immer nach Links - ist Konsitenz sichergestellt.

\subsection{Asymptotic Decider}
- Bilineare Interoplation (an den Flächen entlang?\\
- Fallunterscheidung: Wert von Interpoliertem Kanten-Schnittpunkt vergleichen mit Sattelpunkt-Wert.
- Probleme: Korrketheit der Triangulierung -> Analog zu Marching Cubes, allerdings ist schon Konsitenz sichergestellt - Korrektheit bleibt offen.\\

\pagebreak
\section{Volumenvisualisierung}
Aus Kap 2:
\begin{itemize}[noitemsep]
	\item Schnittflächen mit Einfärbung: Volumen mit einer Ebene schneiden, dann Punkten Farbwert zuordnen.
	\item Jedem Skalar im Volumen optische Eigenschaft zuweisen und auf Projektionsfläche abbilden (Absorptions-, Emissions-Modell, ect.)
	\item Isoflächen (Zeige alle Punkte mit Isowert a an)
\end{itemize}


\subsection{Welche 3D-Volumen-Daten gibt es zu visualisieren?}


\subsection{Schnittflächen}
- Ebne = Position und Normalenvektor\\
- MC-Verfahren\\

\noindent Alternativ: 
- Schnitt entlang einer gekrümmten Ebene (z.B. Vene verfolgen) = Curved Planar Reformation, CPR\\

\noindent Dabei Generell:\\
- Thresholding einsetzbar\\
- Regionenwachstum mit Saatpunkt


\subsection{Durchleuchtung}
Verfahren Beschreiben\\
- ...\\
- Dabei maßgebend Houndsfield-Skala\\

\noindent Emissionsmodell\\

\noindent Absorptionsmodell\\

\noindent Kombination beider Modelle\\

\noindent Absorptionsmodell\\

\noindent Darstellung des resultierenden Skalarfeldes\\
- Mensch schlechte Tiefenwahrnehmung, auf Wahrnehmen von Obeflächen trainiert $^\rightarrow$ deswegen Einsatz Transferfunktion

\subsection{Beleuchtung von Oberflächen}
Notwendig, um 3D sichtbar zu machen.\\

\subsection{Transferfunktionen}
- Color Maps\\
- Color, oppacity (Zahn-Bsp. von Durchleuchtung)\\
- Gradientenberechnung für deutlichere Übergänge -> Baumstamm / Luft vs Blatt vs Luft

\subsection{Sampling-Methoden: Arten der Strahlenverfolgung}
...


\subsection{Methoden nach dem Motto ''Wir machen was anderes als die Physik''}
...

\pagebreak
\section{Skalarfeldtopologie}

\pagebreak
\section{Merkmalskurven}
- Stromlinien, Pfadlinien\\
- Runge Kutte Verfahren

\pagebreak
\section{Vektorfelder}

\pagebreak
\section{Texturbasierte Techniken}

\pagebreak
\section{Vektorfeld-Topologie}

\pagebreak
\section{Merkmalsbestimmung und Wirbelextraktion}


\pagebreak
\section{FRAGEN}
- Bug im Verfahren zu MC (Bei Verbindung über mehrere Zellen) -- was ist der Fehler, wie kann man ihn beheben?\\
- Korrekte Triangulierung = Fehlerbehebung unabh. von Echtwelt?\\ 
- Konsistente Triangulierung = Echtwelt?\\
- Regionenwachstum mit Saatpunkt in Schnittvisualiserung - BSP?\\


\end{document}
