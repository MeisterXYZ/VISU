\documentclass{article}

\title{\vspace{-5mm}%
	\fontsize{24pt}{10pt}\selectfont
	\textbf{VISU- Zusammenfassung}
}	
\usepackage{enumitem}
\usepackage[utf8]{inputenc}

\begin{document}
\maketitle	
\pagebreak

\tableofcontents
\pagebreak


\section{Aufgaben der Visualisierung}
Begriffsdefinition: Drei Zitate zu Visualisierung\\
Aufgaben der Visualisierung Zusammenfassung\\
Aufgaben der Visualisierung I - III\\
Schritte Idealisiertes Vorgehen\\
Ein Wort zu VTK
\pagebreak

\section{Datenformate und Gittermodelle}
\subsection{Eigenschaften von Daten und Datensätzen - Was bilden wir ab?} 
\textbf{Aufbau Datensatz}\\

\begin{itemize}[noitemsep]
	\item Definitionsmenge
	\item Topologie
	\item Funktion $Definitionsmenge \rightarrow Bildmenge$ wobei Bildmenge Attributen der Daten gleichen muss.
\end{itemize}


\noindent \textbf{5 Kriterien nach denen man Datensätze unterscheidet:}
\begin{itemize}[noitemsep]
	\item Dimension des Beobachtungsraumes
	\item Struktur der Definitionsmenge
	\item Struktur der Nachbarschaftsrelation
	\item Dimension der Nachbarschaftsrelation
	\item Wertebereiche der Funktionen
\end{itemize}

\noindent Typen von Daten:
\begin{itemize}[noitemsep]
	\item ....
\end{itemize}
Nominal bis Kontinuierlich
\begin{itemize}[noitemsep]
	\item ....
\end{itemize}

\noindent \textbf{Anforderungen ans Visualisierungssystem}
\begin{itemize}[noitemsep]
	\item effiziente Daten-Speicherung
	\item effizienter Datenzugriff
	\item Abbildbarkeit (?)
	\item Einfachheit (?)
\end{itemize}

\noindent \textbf{Begriffsdefinitionen}\\
\begin{itemize}[noitemsep]
	\item Stetigkeit (Kurz in Umgangssprache aufschreiben)
	\item Topolgie, Kodimension...
\end{itemize}

\noindent \textbf{Das Problem, dass alles was gemessen wird, diskrete Werte sind}\\
Infos auf Folie?\\
Wie man zwischen Werten interpoliert.\\

\noindent Shepard Interpolation\\
$\rightarrow$ Inverse Distanzgewichtung.\\

\noindent Modifizierte Shepard Interpolation\\
$\rightarrow$ Mit Radius

\subsection{Interpolation generell - Alles im Zwischenraum visualisieren können }
Zwischen welchen Werten man interpoliert.\\

\noindent \textbf{Auf der Linie}\\

\noindent \textbf{Im linearen Dreieck}\\

\noindent \textbf{Dreieck mit Baryzentrischen Koordinaten}\\

\noindent \textbf{Im linearen Tetraeder}\\
- mit Baryzentrischen Koordinaten\\
- mit linearem Ansatz\\

\noindent \textbf{Im bilinearen Rechteck}\\
- Entlang der Kanten linear\\
- Innerhalb des Rechtecks Bi-Linear\\

\noindent \textbf{Im linearen Quader}\\

\noindent \textbf{Im linearen Prisma}\\

\noindent \textbf{Neares Neighbour in Voroni-Diagrammen für Scattered Data}\\
$\rightarrow$ Entspricht Shepard Interpolation mit Gewichtungsfaktor für Distanz p gegen Unendlich.
	
\subsection{Gitter - den gesamten Raum visualisieren können}
\noindent \textbf{Was ist ein Gitter?}\\
- Bisher Set an Interpolations-Möglichkeiten\\
- Jetzt ganzen Raum mit Gitter auskleiden - befähigt uns jeden Punkt im Raum abzubilden.\\

\noindent \textbf{Welche Gitter gibt es?}\\

\noindent \textbf{Wie speichert man die Dinger?}\\
- Feste X und Y Delta\\
- Suchstrukturen nennen können. (R-Baum, weitere..)
-- kD-Baum (k-Dimensionaler Baum) - bsp:\\
--- BSP = binary space partitioning Tree
--- Für 2D:  State of the art: Quad-Trees (Ich lege ein gitter drüber, teile das Bild in 4 Teile. So lange bis ich das auf Pixel-Größe habe )
--- Für 3D: Oct-Trees
\pagebreak

\section{Skalare und Skalarfelder}

\subsection{Definifion Feld und Skalarfeld}
Feld: ...\\
Skalarfeld...

\subsection{Darstellungsformen von 2D-Definitionsmenge}
\begin{itemize}[noitemsep]
	\item Einfärbung
	\item Höhenfelder
	\item Isolinien
\end{itemize}

\subsection{Darstellungsformen von 3D-Definitionsmenge}
\begin{itemize}[noitemsep]
	\item Schnittflächen mit Einfärbung: Volumen mit einer Ebene schneiden, dann Punkten Farbwert zuordnen.
	\item Jedem Skalar im Volumen optische Eigenschaft zuweisen und auf Projektionsfläche abbilden (Absorptions-, Emissions-Modell, ect.)
	\item Isoflächen (Zeige alle Punkte mit Isowert a an)
\end{itemize}

\subsection{Color-Maps}
Aspekte raussuchen.

\subsection{Isolinien und Flächen}
- Was sind das?\\
- Interpolation von Isowert in Dreiecks-Flächen (Wo schneidet ein Dreieck die 0-Höhenlinie?)\\
- Interpolation von Isowert in Vierecks-Flächen\\
- Interpolation von Isowert in Polygone -> In Drei-oder Vierecke zerlegen, dann weiter\\

\subsection{Isolinien und Flächen im Raum: Marching Cubes}
- Prinzip\\
- Bug im Verfahren (Bei Verbindung über mehrere Zellen)\\
- Korrekte Triangulierung\\ 
- Konsistente Triangulierung -> Über Festlegung: Bei Sattelpunkt: Liegt immer nach Links - ist Konsitenz sichergestellt.

\subsection{Asymptotic Decider}
- Bilineare Interoplation (an den Flächen entlang?\\
- Fallunterscheidung: Wert von Interpoliertem Kanten-Schnittpunkt vergleichen mit Sattelpunkt-Wert.
- Probleme: Korrketheit der Triangulierung -> Analog zu Marching Cubes, allerdings ist schon Konsitenz sichergestellt - Korrektheit bleibt offen.\\

\pagebreak
\section{Volumenvisualisierung}
Aus Kap 2:
\begin{itemize}[noitemsep]
	\item Schnittflächen mit Einfärbung: Volumen mit einer Ebene schneiden, dann Punkten Farbwert zuordnen.
	\item Jedem Skalar im Volumen optische Eigenschaft zuweisen und auf Projektionsfläche abbilden (Absorptions-, Emissions-Modell, ect.)
	\item Isoflächen (Zeige alle Punkte mit Isowert a an)
\end{itemize}


\subsection{Welche 3D-Volumen-Daten gibt es zu visualisieren?}


\subsection{Schnittflächen}
- Ebne = Position und Normalenvektor\\
- MC-Verfahren\\

\noindent Alternativ: 
- Schnitt entlang einer gekrümmten Ebene (z.B. Vene verfolgen) = Curved Planar Reformation, CPR\\

\noindent Dabei Generell:\\
- Thresholding einsetzbar\\
- Regionenwachstum mit Saatpunkt


\subsection{Durchleuchtung}
Verfahren Beschreiben\\
- ...\\
- Dabei maßgebend Houndsfield-Skala\\

\noindent Emissionsmodell\\

\noindent Absorptionsmodell\\

\noindent Kombination beider Modelle\\

\noindent Absorptionsmodell\\

\noindent Darstellung des resultierenden Skalarfeldes\\
- Mensch schlechte Tiefenwahrnehmung, auf Wahrnehmen von Obeflächen trainiert $^\rightarrow$ deswegen Einsatz Transferfunktion

\subsection{Beleuchtung von Oberflächen}
Notwendig, um 3D sichtbar zu machen.\\

\subsection{Transferfunktionen}
- Color Maps\\
- Color, oppacity (Zahn-Bsp. von Durchleuchtung)\\
- Gradientenberechnung für deutlichere Übergänge -> Baumstamm / Luft vs Blatt vs Luft

\subsection{Sampling-Methoden: Arten der Strahlenverfolgung}
- Bei Strahlenverfolgung entlang der Achsen: Levoy: Farbwert von Voxel am Ende des Voxels = Farbwert zu beginn des voxels + Farvbe des Vooxels + Oppacität des Voxels
- Nicht Entlang der Achsen: Interpolation notwedig
Generell paar Parameter:
- Abtastrate  (Größe des Gitters)\\
--Richtlinie\\
--Theorem aus Signalverarbeitung\\
- Sampling entlang Objekt (Transfpormatio der Strahlen) oder Sampling einfach grade durch\\
- Bessere Daten durch Vorverarbeitung\\
- Abtastung mit Zufallswerten - Verschieben der regulären Abtastpunkte zufällig.\\

Wie gehe ich mit den erhaltenen Punkten um?:

\subsection{Prä- vs. Post-Klassifikation vs. Preintegrated Volume-Rendering}
PräKlassifierung: Erst Eckpunkten Farbe zuweise, dann zwischen Farbwerten interpolieren. (Hier Gefahr des Color bleedings)\\
Postklassifizierung: Erst zwischen Eckpunkten interpolieren, dann in Farbwert übersetzen.\\
Preintegrated Coloring = Integral zwischen Abtastpunkten speichern = Berücksichtigung aller (interpolierten) Zwischenwerte  (kein Color bleeding) \\


\subsection{Non-photorealistic Rendering / abstraktes Rendering }
= Methoden nach dem Motto ''Wir machen was anderes als die Physik'' \\
Entspricht nicht Levoy-Ansatz. Begriff = Stilisierte künstlerisch abstraktes rendering. Für Darstellung des wesentlichen mit zerchnerischer Technik.\\

\noindent \textbf{Silluette Rendering}\\
Kanten des Objektes besoders hervor bei Knicken oder Materialgrenzen.\\

\noindent \textbf{Tone Shading}\\
Schattierung mit einer Farbe: Von nem runden Objekt werden die Vorder- und Rückseite damit klar. Damit kann man 3D geut wahrnehmen. Durch Wahrnehmungstheorie: Warmer ton vorne, kalter Farbton im Schatten.\\

\noindent \textbf{Cartoon Shading}\\
Farbverläufe in Bereiche gleicher Farben unterteilen - Diskretisieren.\\


\subsection{Parallelität ausnutzbar}
Parallelität ausnutzbar weil SIMD.\\
Aufteilung der Daten via kd-Baum (k-Dimensionaler Baum) - halbieren, halbieren ... Baumstruktur. Werte an Rändern dann dublizieren (damit in beiden Teilbäumen mit den Daten gerechnet werden kann) und beim zusammenführen nur 1x werten.


\pagebreak
\section{Skalarfeldtopologie}

\subsection{Definition Skalarfeldtopologie}
- Rubber-Band Geometry\\
- Homöomorphismus \\


- Simplex = Einfache wir Punkt, Limnie Dreieck\\
- Simplizialkomplex = Komplexe Struktur aufgebaut aus Sinmplexen\\
- Triangulierung von nem topologischen Raum X ist ein Simplizialkomplex k zusammen mit nem Homöomirphismus k nach x (wir müssen sagen: Wo gehören die Dreicke hin? Homöomorphismus lässt uns die triangulierten Werde in den zugrundeliegenden echten Raum transformieren.)

\subsection{Morsetheorie}
- Wir haben ne Mannigfaltigkeit (Bspw. Erd-Kugel) darauf haben wir ne differenzierbare Funktion (die Bspw. Höhenfeld darstellt). Mithilfe der Morse-Theorie können wir diese Funktion analytisch untersuchen.

Kritische Punkte und kritische Werte:\\
-- Wenn totale Ableitung gleich 0\\
-- Partielle Ableitung entsprechend auch\\
- Kugel rollt nicht weg\\
--Hoch\\
--Tief\\
--Sattel\\

In Hesse-Matrix stehen alle Kombinationen aller partiellen Ableitungen bis 2. Grades drin.
Für alle kritischen Punkte Hesse-Matrix aufstellen. Darüber Erkenntnis gewinnen, wie es um die Punkte drum rum aussieht (Wo geht's wie starkt hoch und runter?) Daraus können wir die Umgebung aufbauen.

Anwendung auf realistische BSP: Grade Fläche würde überall kritische Punkte liefern, deswegen Dreieck leicht ankippen (kleines Epsilon an Ecken addieren.)

\subsection{Konturbaum}
- Lokales MIN\\
- Lokales MAX\\
- Verbindungs und Trennpunkt\\

- Äquivalente Isokonturen = Zwei Linien untereinander (auf verschiedener Höhe) zwischen denen kein MIN, MAX, Trenn/VerbindPunkt liegen.\\

- Konturbaum Superecke und Superkante.\\

- Vergrößertert Konturbaum (?)\\
- Höhengraph (mit Werten)\\
- Teilgrph\\
- Spilt-tree und join-tree\\


\pagebreak
\section{Vektorfelder}
\subsection{Definition}
Jedem Vektor des Definitionsbereiches/der Domaine ist ein Vektor zugeordnet.

\begin{itemize}[noitemsep]
	\item Zeitunabhängig = Stationär	
	\item Zeitabhängig
\end{itemize}

\subsection{Informationen eines Vektors über Jacobi-Matrix gewinnen}
Leitet man ein Vektorfeld ab (totales Differential) erhält man die Jacobi-Matrix. Diese beschreibt also die Änderung des Vektors in alle Richtungen.\\

\subsection{Kurven im Vektorfeld}
Kurve = Position, die sich über die Zeit ändert.\\
Im Diagramm hätte man beispielsweise ne Kurve in einem 2D-Koordinatensystem. Die Kurve hat nen Start- und nen Zielpunkt. Die Kurve verbindet auf c1-stetige weise diese Punkte. Das ist die Theorie.\\

\noindent In der Praxis fährt zB. ein Auto auf einer 2D-Landschaft. Dann besteht die Kurve aus diskreten, zeitabhängigen Positionspunkten. Dort wo man schneller gefahren ist gibt es weniger Abtastpunkte (liegen die Punkte weiter auseinander). \\

\noindent  $\rightarrow$ Kurvenintegral 1. Art\\ 
Dabei ist F ein Skalarfeld. Setzt man das auf 1 liefert das Kurvenintegral 1. Art die Weglänge. 
Dabei liefert die Ableitung der Kurve nach der Zeit die Richtung der Änderung in einem Punkt, wobei die Länge des Vektors der Geschwindigkeit entspricht.\\

\noindent  $\rightarrow$ Kurvenintegral 2. Art\\ 
Wir haben hier ein Vektorfeld v. Das Vektorfeld kann man sich als Krafteinwirkung vorstellen, die auf das Teilchen, das die Kurve malt, Kraft auswirkt (An dem Teil Arbeit verrichtet). Kurvenintegral 2. Art summiert jetzt die verrichtete Arbeit.\\

\noindent Bildet man das Ringintegral (Integral für geschlossene Kurve, dann nennt man den Wert Zirkulation) in einem Vektorfeld, in zB. dem alle Vektoren parallel zur X-achse liegen und nach Rechts zeigen, ist der Wert des Integrals 0.\\
Mit dem Fahhrad im Kreis fahren wenn Wind stets von Osten kommt, dann ist in die eine Richtung zu treten, in die andere nicht. Wenn der Wind orthogonal zum Fahrrad steht hat das keine Vektorfeld keine Einwirkung.\\

Tornado vs. Windkanal.\\

\noindent Wenn wir bei nem Ringintegral nicht den Wert 0 erhalten (Die Zirkulation ungleich 0 ist), dann liegt ein Wirbel vor.


\noindent  Wir können von nem Ringintegral Teilbereiche Berechnen (Abkürzungen beim im Kreis fahren nehmen). In Summe ergibt das wieder den vollen Kreis.\\
 
\noindent  Wenn wir nen Ring nehmen und darüber die Zirkulation berechen, dann den Ring immer kleiner machen, dann erhalten wir die Berechnung des Rotationswertes für einen Punkt.\\
 
\noindent  Für ein Objekt im 2D-Raum können wir über unendlich vielen Vierecken das Objekt abbilden. Darüber erhalten wir dann das Ringintegral und damit die Zirkulation von dem Objekt.\\
 Über den Satz von Stokes erhaten wir damit auch die Rotation (im Inneren) des Objektes. (Über das Verhalten am Rand, bekommen wir die Info über das Verhalten im Inneren des Objektes).\\


\subsection{Fluss, Quellstärke, Divergenz, ...}

Fluss eines Vektorfeldes gibt die Stärke des Durchflusses durch eine in das Vektorfeld hineingelegte Hyperebene an.\\

\noindent Quellstärke für ein definiertes Volumen in einem Vektorfeld: Ist also die Anzahl an Teilen, die mehr rausgehen als reingehen.\\

\noindent Divergenz	= Quellstärke eines infinitesimalen Quaders.\\

\noindent Durch Satz von Gauß kann man Volumenintegrale durch Flächenintegrale ersetzen.\\

\noindent Divergenzfrei (Divergenz = für infintesimale Einheiten) = Quellenfrei (Begriff für Volumen)\\
\noindent Rotationsfrei (Für Punkte) = Wirbelfrei (Für Felder) \\




\pagebreak
\section{Merkmalskurven}
\subsection{Die vier Merkmalskurven instationärer VF}
In Instationären Vektorfeldern gibt es vier Arten von Merkmalskurven:

\begin{itemize}[noitemsep]
	\item \textbf{Stromlinien} - Momentaufnahme meines instationären Vetorfeldes.\\
	\item \textbf{Pfadlinien} -  Trajektorie eines masselosen Partikels: Ich lasse ein Teil irgendwo im Raum los und überlasse es dem Vektorfeld, der Weg der vom Teilchen genommen wird ist die Pfadlinie.\\
	Berechnung über\\
	- Eulerverfahren - Einschrittverfahren\\
	- Modifiziertes Eulerverfahren (1/2 schritt in die Richtung, dann Vektorfeld an der Position abtasten und daraus nächste Richtung bestimmen)\\
	- Runge-Kutta Verfahren der vierten Ordnung\\
	\item \textbf{Streichlinien} - \\
	wir verfolgen eine Position und werfen dort über die Zeit mehrere Partikel rein.\\
	Hat sich das Feld geändert, nehmen die Teilchen verschiedene Pfade.	(Das ist nicht die Pfadlinie.)\\
	Bsp: Schornstein. Rauch-Wolken-Teil ganz oben ist mit Sicherheit nicht die Linie gegangen, die die gebogene Rauchsäule unten drunter bildet.\\
	
	\item \textbf{Zeitlinien} - Verfolgung einer zusammenhängenden Linie an Tracern über die Zeit.
\end{itemize}


\subsection{Übung Pfadlinien/ Partikel-Trajektion}
- Euler, mod Euler, RK4 üben.\\

\subsection{Terminierung}
Wann hören wir auf, Linien zu zeichnen?\\

\subsection{Anzeigen der Linien}
\textbf{Streamtubes}\\
- Röhrendicke und Farbe verwenden, um Atribute abzubilden.\\
-- Druck\\
-- Geschwindigkeit (hätten wir bei einzelnen Linien nicht) (dünner wenn schneller $\rightarrow$ darf aber nicht zu dünn werden, wenn es dann nicht mehr da ist die Linie)\\
-- Zeit\\

\noindent \textbf{Streamribbons}\\
- Drehung des Partikels um sich selbst lässt sich durch Drehung eines solchen Bandes darstellen\\
- mehrere Bänder können auch einen gemeinsamen Wirbel ergeben\\
- Größen und Farbkodierung analog zu Röhren\\

\noindent \textbf{Stromlinien-Platzierung}\\
- Man braucht ein paar Stromlinien, um den Datensatz zu verstehen. Aber: keine Überschneidungen, keine Häufungen\\
Es gibt Bereiche, in denen Stromlinien zusammenlaufen. Da will man nicht alle Stromlinien zeichnen.\\
Erste Möglichkeit das anzugehen:
Das ist heftig:  Räumliche Suchstuktur bauen, um zu checken, ob sich Linien zu nahe kommen um ggf. Zeichnen von Linien abzubrechen.

\noindent Noch ein Ansatz:
Starten auf schwarzer Fläche. Stromlinie einzeichnen. Dann verwischen. Bspw. mit Gauß Glockenkurve. Dann nächste Linie zeichnen, dort wo es noch ziemlich dunkel ist. Dann immer weiter, bis ein ziemlich graues Bild entsteht.\\
- Bild mit Schwarzen/weißen Flächen nicht so gut.\\
- Wie gut ne Linie war würde sich zeigen, ob sich die Varianz  des gesamten Bildes verbessert hat.
Meiste Verfahren sind Brute-Force-Verfahren: haufenweise Linien berechnen und prüfen, ob es besser wird.

\pagebreak
\section{Texturbasierte Techniken}
\subsection{Schlieren als Motivation}
Idee: Strömung sichtbar machen.\\
Messtechnik für 2D oder 3D-Strömungen = Schlierenbilder.\\

\noindent Oberfläche des Objektes mit Ölfilm benetzen, dann bilden sich nach Strömung Streifen auf dem Ölfilm.\\
Oder Dichte von Teilchen im Windkanal betrachten.\\

\noindent Aufbau Spiegelvorrichtung für Messen von Schlieren = Strioscopie TODO

\subsection{Spot Noise}
Methode Strömungsdaten darzustellen.\\
Spot = Farbklecks.\\
Man wählt zufällig Positionen in der Textur und zeichnet einen Spot an dieser Stelle, der so geformt ist, dass er die Richtung des Vektorfeldes wiedergibt. In Kombination mit Farbabbildungen z.B. des Drucks ergeben sich dann Bilder, die Linienmuster entlang von Stromlinien bilden.\\

\noindent In 3D auch möglich, da allerdings Verdeckungsproblem, daher Verteilung von Spots von anderen Parametern abhängig gemacht.\\

\subsection{DDC}

\noindent Discrete Differential Analyzer-Faltungsverfahren:\\
Hier wird eine Linie in Richtung des Vektorfeldes gerastert und über diese gemittelt.\\
Problem: bei Starker Krümmung einer Linie verwischte Darstellung. \\


\subsection{LIC}
Zum Überwinden des Problemes in DDC LIC verwenden.\\
= Line Integral Convolution oder LinienIntegral-Faltung.\\
= Auto in Windkanal stellen, Ölfilm drauf, Windkanal an, Windkanal aus, Ölfilm auf Oberfläche anschauen. Wenn du das als Algorithmus umsetzt, hast dus. 

\noindent \textbf{LIC}\\

\noindent \textbf{Fast LIC}\\

\noindent \textbf{LIC auf Volumen}

\pagebreak
\section{Vektorfeld-Topologie}

\subsection{Das Ziel}
Untersucht die Frage nach Gesamtbild des Flusses. Also Anfang und Ende von Stromlinien ansehen. Dadurch Bereiche ähnlichen Stromlinienverhaltens identifizieren. Das ist das Ziel.\\

\subsection{Startmenge, Zielmenge, Becken, Topologie}
\noindent \textbf{Startmenge}\\
Für jeden Punkt des Beobachtungsraumes Stromlinie zurückverfolgen. Die Punkte, deren Stromlinienposition am nächsten an minus unendlich dran sind, bilden die Startmenge.\\
Warum so kompliziert? - Wegen Kreis-Problem.\\

\noindent \textbf{Zielmenge}\\
= Startmenge Rückwerts.\\

\noindent \textbf{Becken}\\
Becken eines Punktes = Alle Punkte des Definitionsbereiches, die auf den Linien liegen, die vom gegebenen Punkt ausgehen.\\ 
Man kann auch ein Becken für mehrere Punkte definieren.\\

\noindent \textbf{Vektorfeldtopoliogie}\\
Topologie wird definiert als Zerlegung von Beaobachtungsraum B in Komponenten. Diese Komponenten ergeben Sich aus den Schnitten Von Apha- und Omega-Becken. \\
Eine Komponente besteht aus den Stromlinien, die die gleich Start- und Zielmenge haben.


\subsection{Seperatritzen, kritische Punkte}
\noindent \textbf{Kritische Punkte}\\
Kritische Punkte = Nullstellen des Vektorfeldes. Start- und Zielmenge besteht aus dem Punkt selbst. Das ist eine sog. Singularität.\\
Kritische Punkte kann man durch Nachbarschafts-Betrachtung klassifizieren.\\

\noindent \textbf{Seperatritzen}\\
- Was einen interessiert sind die Teile, die bestimmte Bereiche abtrennen.\\
- Das sind immer Linien die ein Besonderes Verhalten haben. Nämlich die kritische Punkte verbinden. Solche Linien sind die sog. Seperatritzen.\\

\noindent Quellen, Senken, Sattelpunkte\\

\noindent Vergleich zu Skalarfeldern

\subsection{Weiteres?}

\pagebreak
\section{Merkmalsbestimmung und Wirbelextraktion}
Merkmale (engl. Features) = 


Hängen bleiben sollte:\\
- Versch. mathematische Modelle zum Beschreiben der Wirbel\\
- Ideen hinter den Algorithmen zum finden \\
- Was ist ein Merkmal, was ist interessant, verfolgen über Zeit\\
- Verschiedene Ideen dazu das anzuzeigen

\pagebreak
\section{FRAGEN}
- Bug im Verfahren zu MC (Bei Verbindung über mehrere Zellen) -- was ist der Fehler, wie kann man ihn beheben?\\
- Korrekte Triangulierung = Fehlerbehebung unabh. von Echtwelt?\\ 
- Konsistente Triangulierung = Echtwelt?\\
- Regionenwachstum mit Saatpunkt in Schnittvisualiserung - BSP?\\
- Was versteht man unter Resampling?\\
- Texturbasiertes Rendering?\\
- Bsp. für Kurvenintegral 2. Art\\
- Skalarpotential?


\end{document}
